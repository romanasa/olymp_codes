\section{Задача Е. Города}

\begin{frame}[t]{Задача Е. Города}

  \vspace{2cm}
  \begin{center}
    \LARGE Задача Е. Города
  \end{center}
\end{frame}


\begin{frame}[t]{}
  \vspace{3cm}

  \begin{itemize}
    \item Разбор задачи~--- Роман Коробков
  \end{itemize}
\end{frame}



\subsection{Постановка задачи}

\begin{frame}[t]{Постановка задачи}
\begin{itemize}
    \item Дано взвешенное дерево $T$, в котором $n$ листьев, и степень каждой внутренней вершины $ \ge 3$.
    \item За один запрос можно узнать $d(u, v)$ $-$  кратчайшее расстояние между $u$ и $v$.
    \item Нужно найти центры дерева, а также сказать является какой-либо центр медианой для листьев (при его удалении количество листьев
    	в каждой из компонент не превышает $n / 2$).
    \item Необходимо сделать это за $ \lceil \frac{7N}{2} \rceil$ запросов.
\end{itemize}
\end{frame}

\subsection{Решение задачи}

\begin{frame}[t]{Радиусы и центры}
\begin{itemize}
	\item Пусть $v$ - вершина с номером 1. Найдем вершину $s$ с наибольшим расстояние от $v$, 
	$t$ с наибольшим расстоянием от $s$. $d(s, t)$ $-$ диаметр, а искомые центры лежат на пути от $v$ до $s$.
	\item $f(u) = (d(s, u) + d(s, v) - d(u, v)) / 2 $ $-$ расстояние от $s$ до вершины $x$, лежащей на пути $sv$.
	\item Вершины $u$, для которых $\max(f(u), diam - f(u))$ минимально $-$ центры.
	\item Таким образом за $2 \cdot n - 3$ запроса можно найти радиус дерева $r$.
\end{itemize}
\end{frame}


\begin{frame}[t]{Решение задачи}
	\begin{itemize}
		\item Пусть $X = \{ u $ $|$ $d(s, u) + d(s, v) - d(u, v) = 2 \cdot r \}$. $X$ - множество листов, которые подвешены к центру.
		\item Если $ d(s, x_1) + d(s, x_2) - d(x_1, x_2) > 2 \cdot r $, то $x_1$ и $x_2$ будут находиться в разных компонентах связности, после удаления центра.
		\item Пусть $R(a, b) = 1$, если $a$ и $b$ лежат в одной компоненте, и $0$ иначе.
		\item Переформулируем нашу задачу: есть $n$ цветных шаров, необходимо узнать есть ли среди них более $n/2$ шаров одного цвета, используя функцию $R$. 
	\end{itemize}
\end{frame}
	
	

\begin{frame}[t]{Решение задачи}
	\begin{itemize}
		\item Научимся решать такую задачу за $\lceil \frac{3N}{2} \rceil $ запросов.
		\item Заведём $A$, $B$.
		\item Рассмотрим элементы в любом порядке. Если текущий элемент равен последнему элементу в $A$, то добавим его в $B$. Иначе добавим его в $A$, а затем последний элемент из $B$ переложим в $A$.
		\item Заметим, что все элементы $B$ равны последнему элементу $A$, и только этот элемент может встречать более $n/2$ раз. Пусть его цвет $c$
	\end{itemize}
\end{frame}


\begin{frame}[t]{Решение задачи}
	\begin{itemize}
		\item Затем будем делать следующее, пока в $A$ есть элементы:
		\item Если в $A$ последний элемент имеет цвет $c$, то удалим два последних элемента из $A$. Иначе удалим последний элемент из $A$ и $B$.
		\item Если после всех операций в $B$ остались элементы, то цвет $c$ встречается более $n/2$ раз.
		\item Суммарное количество запросов $(n - 1) + (n - 2) + n + \lceil \frac{n}{2} \rceil < \lceil \frac {7n}{2} \rceil $
	\end{itemize}
\end{frame}
	