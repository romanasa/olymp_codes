\section{Задача D. Полет мечты}

\begin{frame}[t]{Задача D. Полет мечты}

  \begin{center}
    \LARGE Задача D. Полет мечты
  \end{center}
  \begin{center}
	  \includegraphics[width=7cm]{pics/dreamrun.jpg}
  \end{center}
\end{frame}

\begin{frame}[t]{}
  \vspace{3cm}
  \begin{itemize}
    \item Идея задачи --- Антон Гардер, Антон Евдокимов
    \item Подготовка тестов --- Антон Гардер
    \item Разбор задачи --- Антон Гардер
  \end{itemize}
\end{frame}

\subsection{Постановка задачи}

\begin{frame}[t]{Постановка задачи}
Входные данные:
\begin{itemize}
    \item Дана точка на поверхности Земли.
    \item Нужно переместиться из неё на $d$ км. на юг, на запад и на север. Как итог вернуться в начало.
    \item Нельзя подлетать близко к южному полюсу, $d \ge 1$.
    \item Найти подходящее $d$.
\end{itemize}
\end{frame}

\subsection{Решение задачи}

\begin{frame}[t]{Решение задачи}
\begin{itemize}
    \item Если старт~--- северный полюс, то $d$~--- любое.
    \item Иначе, наш путь выглядит следующим образом:
    \begin{itemize}
        \item Спускаемся на юг на $d$ километров в точку $t$;
        \item Пролетаем $d$ километров вокруг Земли и возвращаемся в точку $t$;
        \item Поднимаемся на север на $d$ километров из точки $t$ обратно на старт.
    \end{itemize}
    \item Как найти $d$?
\end{itemize}
\end{frame}

\begin{frame}[t]{Решение задачи}
\begin{itemize}
    \item Если старт выше экватора, спустимся сначала до экватора.
    \item Начинаем спускаться вниз.
    \item Длина пройденного пути вниз увеличивается, а длина окружности,
по которой мы будем совершать оборот вокруг Земли, уменьшается.
    \item Двоичным поиском находим точку, в которой эти длины совпадут.
    \item Длина пути от старта до этой точки~--- искомое $d$.
    \item Ограничения в задаче заданы так, что данное решение работает без проблем с точностью.
\end{itemize}
\end{frame}
