\section{Задача Y. Фальшивая монета}

\begin{frame}[t]{Задача Y. Фальшивая монета}

  \begin{center}
    \LARGE Задача Y. Фальшивая монета
  \end{center}

  \begin{itemize}
    \item Идея задачи --- Фольклор
    \item Подготовка тестов --- Николай Ведерников
    \item Разбор задачи --- Дмитрий Якутов, Николай Будин
  \end{itemize}
\end{frame}

\subsection{Постановка задачи}

\begin{frame}[t]{Постановка задачи}
\begin{itemize}
    \item Есть $12$ монет, среди них одна фальшивая, не равная по весу остальным.
    \item Нужно не более, чем за $3$ взвешивания на чашечных весах без гирь, определить фальшивую.
\end{itemize}
\end{frame}

\subsection{Решение задачи}

\begin{frame}[t]{Решение задачи}
\begin{itemize}
    \item Пронумеруем монеты числами от $1$ до $12$.
    \item Взвесим монеты $\{1, 2, 3, 4\}$ с монетами $\{5, 6, 7, 8\}$.
    \item Рассмотрим 2 случая: они оказались равны или не равны.
\end{itemize}
\end{frame}

\begin{frame}[t]{Решение задачи. Случай 1}
\begin{itemize}
    \item Тогда среди них нет фальшивой, то фальшивая~--- одна из монет $\{9, 10, 11, 12\}$.
    \item Взвесим $9$ и $10$.
    \item Если они оказались равны, фальшивая монета среди $\{11, 12\}$.
    Взвесим $11$ с настоящей, например, $9$, если они равны, фальшивая~--- $12$, иначе~--- $11$.
    \item Если $9$ и $10$ не равны, фальшивая среди них. Взвесив $9$ с любой настоящей, мы поймем, какая из $9$ и $10$ фальшивая.
\end{itemize}
\end{frame}


\begin{frame}[t]{Решение задачи. Случай 2}
\begin{itemize}
    \item Предположим, $\{1, 2, 3, 4\}$~--- тяжелее, второй случай разбирается аналогично. Тогда, либо одна из монет $\{1, 2, 3, 4\}$~--- фальшивая и тяжелее настоящей, 
    либо одна из монет $\{5, 6, 7, 8\}$~--- фальшивая и легче настоящей.
    \item Взвесим $\{1, 2, 5\}$ и $\{3, 4, 6\}$.
    \item Если они равны, среди них нет фальшивой, и фальшивая среди $7$ и $8$. Оставшимся взвешиванием определим её.
\end{itemize}
\end{frame}

\begin{frame}[t]{Решение задачи. Случай 2}
\begin{itemize}
    \item Если $\{1, 2, 5\}$~--- тяжелее, то $5$ не может быть фальшивой, так как в этом случае эта кучка была бы легче второй кучки. Аналогично $3$ и $4$ не могут быть фальшивыми.
    \item Взвесим $1$ и $2$. Если они не равны, то более тяжелая из них~--- фальшивая. Иначе, фальшивая~--- $6$.
    \item Аналогично разберем случай, когда $\{1, 2, 5\}$ легче чем $\{3, 4, 6\}$.
\end{itemize}
\end{frame}