\section{Задача E. Занимательное дежурство}

\begin{frame}[t]{Задача E. Занимательное дежурство}

  \begin{center}
    \LARGE Задача E. Занимательное дежурство
  \end{center}
  \begin{center}
      \includegraphics[width=7cm]{pics/duty.jpg}
  \end{center}
\end{frame}

\begin{frame}[t]{}
  \vspace{3cm}
  \begin{itemize}
    \item Идея задачи --- Анна Малова
    \item Подготовка тестов --- Григорий Шовкопляс
    \item Разбор задачи --- Григорий Шовкопляс
  \end{itemize}
\end{frame}

\subsection{Постановка задачи}

\begin{frame}[t]{Постановка задачи}
\begin{itemize}
    \item Дан набор латинских букв.
    \item Каждый ход:
        \begin{itemize}
             \item Удаляется две одинаковых буквы.
             \item Добавляется любая новая.
        \end{itemize}
    \item Кто не может сделать ход, проигрывает.
    \item Нужно узнать, кто выиграет при правильной стратегии.
\end{itemize}
\end{frame}

\subsection{Решение задачи}

\begin{frame}[t]{Общие наблюдения}
\begin{itemize}
    \item Порядок букв не важен.
    \item Некоторые буквы встречаются в строке чётное количество раз $N_{even}$, а некоторые нечётное $N_{odd}$.
    \item Каждое состояние игры можно охарактеризовать числами $N_{even}$ и $N_{odd}$.
    \item Проигрышная позиция~--- $N_{odd}$ совпадает с числом букв в наборе.
    \item При удалении двух букв четность не меняется.
    \item При добавлении одной меняется.
    \item Игра не содержит циклов.
\end{itemize}
\end{frame}

\subsection{Решение задачи}

\begin{frame}[t]{Динамическое программирование}
\begin{itemize}
    \item Стандартная задача динамического программирования для подсчёта выигрышных позиций.
    \item Состояние ~--- общее число букв в наборе $K$, а также $N_{even}$ и $N_{odd}$.
    \item Переходы:
        \begin{itemize}
            \item $\langle K, N_{even}, N_{odd} \rangle \longrightarrow \langle K - 1, N_{even} - 1, N_{odd} + 1 \rangle$.
            \item $\langle K, N_{even}, N_{odd} \rangle \longrightarrow \langle K - 1, N_{even} + 1, N_{odd} - 1 \rangle$.
        \end{itemize}
    \item Полезное замечание: $N_{even} = 26 - N_{odd}$. Можно отказаться от третьего параметра.
\end{itemize}
\end{frame}

\subsection{Решение задачи}

\begin{frame}[t]{Альтернативное решение}
 Формула:
 \begin{itemize}
     \item $N_{odd} = K$: выигрывает Дима.
     \item $N_{odd}$~--- четное: Гриша выиграет только, если $N_{odd} = K - 2$.
     \item $N_{odd}$~--- нечетное: Дима выиграет только, если $N_{odd} < K - 2$.
 \end{itemize}
 Доказательство формулы следует из предыдущего решения и оставляется слушателю в качестве упражнения.
\end{frame}