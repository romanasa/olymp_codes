\section{Задача N. <<Unknown>>}

\begin{frame}[t]{Задача N. <<Unknown>>}

  \begin{center}
    \LARGE Задача N. <<Unknown>>
  \end{center}

  \begin{itemize}
    \item Идея задачи --- Xxx Yyy
    \item Подготовка тестов --- Xxx Yyy
    \item Разбор задачи --- Xxx Yyy
  \end{itemize}
\end{frame}

\subsection{Постановка задачи}

\begin{frame}[t]{Дискретная задача о рюкзаке}
Входные данные:
\begin{itemize}
    \item Есть $n$ вещей и рюкзак вместимостью $s$
    \item Вещь с номером $i$ характеризуется размером (весом) $w_i$
     и ценой $c_i$
\end{itemize}
Нужно выбрать некоторое подмножество вещей так, чтобы:
\begin{itemize}
    \item Суммарный размер выбранных вещей не превосходил $s$
    \item Суммарная цена выбранных вещей была как можно больше
\end{itemize}
Дополнительное условие: $w_i$ "--- положительные целые числа.
\end{frame}

\subsection{Варианты постановки задачи}

\begin{frame}[t]{Варианты постановки задачи}
\begin{itemize}
    \uncover<1->%
    {%
    \item Нужно набрать как можно больше вещей ($c_i = 1$) \\
    }%
    \uncover<2->%
    {%
     \textit{Решается жадным алгоритмом:
      отсортируем вещи по весу и будем брать, начиная с самой маленькой,
      пока они помещаются}
    }%
    \uncover<3->%
    {%
    \item Цен нет, нужно набрать как можно больший вес ($c_i = w_i$) \\
    }%
    \uncover<4->%
    {%
     \textit{Решается аналогично исходной постановке}
    }%
    \uncover<5->%
    {%
    \item Вещей каждого типа не одна, а сколько угодно \\
    }%
    \uncover<6->%
    {%
     \textit{Решается аналогично исходной постановке}
    }%
\end{itemize}
\end{frame}

\subsection{Пример}

\begin{frame}[t]{Пример}%
\begin{center}%
        $n = 4$ \\
        $s = 6$ \\
        \medskip
    \begin{tabular}{|c|c|c|c|c|}
        \hline
        $i$   & \hl<3,5>{$1$} & \hl<2-3>{$2$} & \hl<3,6>{$3$} & \hl<2,5-6>{$4$} \\
        \hline
        $w_i$ & \hl<3,5>{$1$} & \hl<2-3>{$2$} & \hl<3,6>{$3$} & \hl<2,5-6>{$4$} \\
        \hline
        $c_i$ & \hl<3,5>{$2$} & \hl<2-3>{$3$} & \hl<3,6>{$5$} & \hl<2,5-6>{$7$} \\
        \hline
    \end{tabular}
        \medskip
    \\
    \only<2>%
    {%
        Оптимальное решение: \\ 
        Выберем вещи с номерами $2$ и $4$. \\
        Суммарный вес равен $2 + 4 = 6$. \\
        Суммарная цена равна $3 + 7 = 10$. \\
    }%
    \only<3>%
    {%
        Ещё одно оптимальное решение: \\ 
        Выберем вещи с номерами $1$, $2$ и $3$. \\
        Суммарный вес равен $1 + 2 + 3 = 6$. \\
        Суммарная цена равна $2 + 3 + 5 = 10$. \\
    }%
    \only<4>%
    {%
        Всего есть $2 \cdot 2 \cdot 2 \cdot 2 = 16$ вариантов
        решения: \\
        каждую вещь можно независимо от других либо взять,
        либо не взять. \\
    }%
    \only<5>%
    {%
        Некоторые решения неоптимальны: \\
        Выберем вещи с номерами $1$ и $4$. \\
        Суммарный вес равен $1 + 4 = 5$. \\
        Суммарная цена равна $2 + 7 = 9$. \\
        Заметим, что в этом решении не добавить ещё одну вещь. \\
    }%
    \only<6>%
    {%
        Некоторые решения невозможны: \\
        Выберем вещи с номерами $3$ и $4$. \\
        Суммарный вес равен $3 + 4 = 7 > s$. \\
    }%
\end{center}
\end{frame}

