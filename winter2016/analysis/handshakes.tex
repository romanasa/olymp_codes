\section{Задача F. Рукопожатия}

\begin{frame}[t]{Задача F. Рукопожатия}

  \begin{center}
    \LARGE Задача F. Рукопожатия
  \end{center}
  \begin{center}
	  \includegraphics[width=7cm]{pics/handshakes.jpg}
  \end{center}
\end{frame}

\begin{frame}[t]{}
  \vspace{3cm}
  \begin{itemize}
    \item Идея задачи --- Глеб Евстропов
    \item Подготовка тестов --- Глеб Евстропов
    \item Разбор задачи --- Глеб Евстропов
  \end{itemize}
\end{frame}

\subsection{Постановка задачи}

\begin{frame}[t]{Постановка задачи}
\begin{itemize}
    \item Загадан некоторый граф $G = (V, E)$.
    \item Для каждой вершины $i$ дана её степень в подграфе на первых $i$ вершинах.
    \item Требуется найти максимально возможную $deg(v)$ для подходящего графа.
\end{itemize}
\end{frame}

\subsection{Решение задачи}

\begin{frame}[t]{Решение задачи}
\begin{itemize}
    \item Для каждой вершины $i$ будем узнавать максимально возможную степень в подграфе на вершинах $i, \ldots, n$.
    \item В выделенном подграфе вершина $i$ может иметь рёбра только с вершинами $j$, у которых степень в подграфе на вершинах $1, \ldots, j$ больше нуля.
    \item Так как нас требуют максимизировать степень вершины, мы соединяем вершину $i$ со всеми доступными вершинами.
    \item Для определения числа таких рёбер мы перебираем вершины с конца, подсчитывая количество вершин с ненулевым числом рёбер
    в соответствующем подграфе.
\end{itemize}
\end{frame}
