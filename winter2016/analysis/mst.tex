\section{Задача B. MST Inc.}

\begin{frame}[t]{Задача B. MST Inc.}
	
	\vspace{2cm}
	\begin{center}
		\LARGE Задача B. MST Inc.
	\end{center}
\end{frame}

\begin{frame}[t]{}
  \vspace{3cm}

  \begin{itemize}
    \item Разбор задачи~--- Роман Коробков
  \end{itemize}
\end{frame}

\subsection{Постановка задачи}

\begin{frame}[t]{Постановка задачи}
\begin{itemize}
	\item Дано $n$ событий двух типов.
	\item События первого типа обозначают объявление переменной, а второго --- присваивание некоторого значения переменной.
	\item В каждом событии используется некоторый набор переменных $X_{i}$.
	\item Назовем противоречием повторное объявление переменной или использование необъявленной переменной.
\end{itemize}
\end{frame}


\begin{frame}[t]{Постановка задачи}
\begin{itemize}	
	\item Поступают запросы трех типов.
	\item Изменить $i$-е событие на объявление или использование некоторой переменной.
	\item Изменить набор переменных в $i$-м событии на $X'_{i}$. 
	\item Проверить нет ли противоречий для отрезка событий с индексами от $L$ до $R$. 
\end{itemize}
\end{frame}



\subsection{Решение задачи}

\begin{frame}[t]{Решение задачи}
\begin{itemize}
	\item Разделим появление переменной в коде на два события: объявление и использование.
	\item Нужно проверить все ли переменные объявлены до использования, и нет ли повторного объявления переменной.
	\item Будем отдельно проверять каждое из условий.
\end{itemize}
\end{frame}


\begin{frame}[t]{Объявление переменной}
	\begin{itemize}
		\item Для каждого объявления переменной $x$ в позиции $i$ обозначим за $V_{i}$ индекс предыдущего ее объявления или $-1$, если переменная объявляется первый раз.
		\item Тогда повторное объявление отсутствует, когда $\max_{L \le i \le R}\{V_{i}\} < L$
		\item При изменении событий $V_{i}$ легко пересчитывается, так что можно использовать структуру, которая поддерживает операции изменения в точке и запроса максимума на отрезке.
		
	\end{itemize}
\end{frame}

\begin{frame}[t]{Использование переменной}
	\begin{itemize}
		\item Для каждого использования переменной $x$ в позиции $i$ обозначим за $U_{x}$ индекс предыдущего ее использования(объявления) или $-1$, если переменная используется первый раз.
		\item Тогда необъявленные переменные отсутствуют, когда $\min_{L \le i \le R}\{U_{i}\} \ge L$
		\item При изменении событий $U_{i}$ легко пересчитывается, так что можно использовать структуру, которая поддерживает операции изменения в точке и запроса минимума на отрезке.
		 
	\end{itemize}
\end{frame}

\begin{frame}[t]{Время работы}
	\begin{itemize}
		\item Время работы: $O(n\log n + m\log n)$, где $n$ ~--- количество событий, $m$~--- количество запросов.
	\end{itemize}
\end{frame}
