\section{Задача А. МКС}

\begin{frame}[t]{Задача А. МКС}

  \vspace{2cm}
  \begin{center}
    \LARGE Задача А. МКС
  \end{center}
\end{frame}


\begin{frame}[t]{}
  \vspace{3cm}

  \begin{itemize}
    \item Разбор задачи~--- Роман Коробков
  \end{itemize}
\end{frame}



\subsection{Постановка задачи}

\begin{frame}[t]{Постановка задачи}
\begin{itemize}
    \item Дано дерево. По очереди добавляются ребра.
    \item Можно узнать, есть ли между двумя множествами вершин ребро. 
    \item Сказать, какое ребро добавилось.
\end{itemize}
\end{frame}

\subsection{Решение задачи}

\begin{frame}[t]{Решение задачи}
\begin{itemize}
	\item Пусть мы знаем, что между $A$ и $B$  есть одно ребро.
	\item $F_{A}(m)$ --- существует ли ребро, между первыми $m$ элементами множества $A$ и множества $B$.
	\item Если $F_{A}(m) = 1$, то и $F_{A}(m + 1) = 1$. Бинарным поиском по $m$ найдем вершину $v_{A}$ из $A$, из которой исходит ребро.
	\item Аналогично найдем $v_{B}$. 
	\item Искомое ребро --- $(v_{A}, v_{B})$
\end{itemize}
\end{frame}


\begin{frame}[t]{Решение задачи}
	\begin{itemize}
		\item Пусть на текущем шаге компоненты $C_{0}, C_{1}$, $\ldots, C_{k - 1}$.
		\item За $i$-й запрос узнаем существует ребро между $A = \{$компоненты, в номере которых в $i$-м бите $0\}$, $B = \{$компоненты, в номере которых в $i$-м бите $1\}$.
		\item $1$-й запрос: $A = \{ C_{0}, C_{2}, C_{4}, \ldots \}, B = \{ C_{1}, C_{3}, C_{5}, \ldots \}$.
		\item $2$-й запрос: $A = \{ C_{0}, C_{1}, C_{4}, C_{5}, \ldots \}, B = \{ C_{2}, C_{3}, C_{6}, C_{7}, \ldots \}$.
	\end{itemize}
\end{frame}
	

\begin{frame}[t]{Решение задачи}
	\begin{itemize}
		\item Очевидно, что таким образом мы найдем компоненты, между которыми добавлено ребро. Затем двумя бинарными поисками определим вершины.
	\end{itemize}
\end{frame}
	
	
	\begin{frame}[t]{Решение задачи}
		\begin{itemize}
			\item Тогда общее количество запросов $=$ $\sum\limits_{i = 2}^n { \lceil\log_2{i}\rceil + \lceil\log_2{i}\rceil + \lceil\log_2{i} - 1}\rceil $.
			\item На практике максимальное число запросов $1613$.
		\end{itemize}
	\end{frame}
