\section{Задача Z. Угадай строку}

\begin{frame}[t]{Задача Z. Угадай строку}

  \begin{center}
    \LARGE Задача Z. Угадай строку
  \end{center}

  \begin{itemize}
    \item Идея задачи --- Дмитрий Филиппов
    \item Подготовка тестов --- Дмитрий Филиппов
    \item Разбор задачи --- Николай Будин
  \end{itemize}
\end{frame}

\subsection{Постановка задачи}

\begin{frame}[t]{Постановка задачи}
\begin{itemize}
    \item Жюри загадало битовую строку длины $n$ ($1 \le n \le 1000$).
    \item Программа может делать запросы вида: <<Присутствует ли битовая строка $s$ как подстрока в загаданной?>>
    \item Нужно отгадать строку не более чем за $1024$ запроса.
\end{itemize}
\end{frame}

\subsection{Решение задачи}

\begin{frame}[t]{Решение задачи}
\begin{itemize}
    \item Двоичным поиском найдем максимальную по длине строку из нулей, входящую в загаданную. Обозначим за $k$ ее длину.
    \item Теперь будем дописывать к ней справа единицы и задавать вопрос про новую строку.
    \item Если такая строка присутствует, допишем в конец новую единицу.
    \item Если такой строки нет, заменим последнюю единицу на ноль, предполагая, что такая подстрока в строке есть, и допишем единицу.
\end{itemize}
\end{frame}

\begin{frame}[t]{Решение задачи}
\begin{itemize}
    \item Заметим, что мы не проверяем принадлежность строки после замены последней единицы на ноль. После того, как получим суффикс загаданной строки, мы начнем всегда дописывать нули в конец.
    \item Когда в конце нашей строки нулей станет больше $k$, наша строка гарантированно не будет являться подстрокой загаданной.
    \item Двоичным поиском найдем префикс нашей строки, который является суффиксом загаданной.
    \item Теперь будем дописывать слева единицы таким же образом, пока длина строки не станет равна $n$.
\end{itemize}
\end{frame}
