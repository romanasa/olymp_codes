\documentclass[10pt,a4paper,twocolumn,landscape,oneside]{article}

\usepackage[T2A]{fontenc}
\usepackage[utf8]{inputenc}                                                          
\usepackage[english,russian]{babel}
\usepackage{expdlist}
\usepackage{form}
\usepackage{amsmath}
\usepackage{amssymb}

\binoppenalty=10000
\relpenalty=10000
\sloppy

\providecommand{\reach}{\rightsquigarrow}
\renewcommand{\t}{\texttt}

\raggedbottom

\begin{document}

Фамилия \scriptline{5cm} Имя \scriptline{5cm}


\begin{enumerate}

\item 
Выберите истинные утверждения.

\begin{tabular}{cp{12cm}}
\pchoice{Если объекты равны, то хеши могут быть равны.}
\pchoice{Если объекты равны, то хеши могут быть не равны.}
\pchoice{Если хеши равны, то объекты могут быть равны.}
\pchoice{Если хеши равны, то объекты могут быть не равны.}
\pchoice{Если объекты не равны, то хеши всегда не равны.}
\pchoice{Если хеши не равны, то объекты всегда не равны.}
\pchoice{$h(x)=1$ является корректной хеш-функцией.}
\pchoice{$h(x)=\operatorname{rand}\{0, 1\}$ является корректной хеш-функцией.}
\end{tabular}

\item
Для хеширования строки $s$, состоящей из маленьких латинских букв, была 
применена формула $\sum\limits_{i=1}^{n}(s_i - \texttt{‘a’}) 7^{n-i}$. 
Приведите пример двух различных строк, имеющих одинаковый хеш.


\item
Выберите истинные утверждения.

\nopagebreak
\begin{tabular}{cp{12cm}}
\pchoice{Алгоритм Ахо-Корасик позволяет найти количество вхождений строк $S_i$ в текст $T$ за время $O(\sum |S_i|+|T|)$.}
\pchoice{Алгоритм Ахо-Корасик позволяет найти все вхождения строк $S_i$ в текст $T$ за время $O(\sum |S_i|+|T|)$.}
\pchoice{Алгоритм Ахо-Корасик позволяет найти наибольшую общую подстроку двух строк с длинами $n$ и $m$ за $O(n+m)$.}
\pchoice{Алгоритм Ахо-Корасик позволяет найти наибольший префикс строки $S$ входящий в строку $T$ за $O(|S|+|T|)$.}
\pchoice{Глубина вершины в которую указывает суффиксная ссылка, построенная алгоритмом Ахо-Корасик для одной строки, совпадают с префикс функцией для этой строки.}
\pchoice{Суффиксные ссылки, построенные алгоритмом Ахо-Корасик для одной строки, совпадают с $Z$-функцией для этой строки.}
\end{tabular}

\item
Пусть $Z[i]$ - $Z$-функция, а $\pi$ - префикс функция.
Выберите истинные утверждения.

\nopagebreak
\begin{tabular}{cp{12cm}}
\pchoice{Для строки $S$ и любого $i < j$ верно, что $Z[i] \le Z[j]$.}
\pchoice{Для строки $S$ и любого $i < j$ верно, что $Z[i] \ge Z[j]$.}
\pchoice{Для строки $S$ и любого $i < j$ верно, что $\pi[i] \le \pi[j]$.}
\pchoice{Для строки $S$ и любого $i < j$ верно, что $\pi[i] \ge \pi[j]$.}
\pchoice{Для строки $S$ и любого $i$ верно, что $\pi[i] \le Z[i]$.}
\pchoice{Для строки $S$ и любого $i$ верно, что $\pi[i] \ge Z[i]$.}
\pchoice{Для строки $S$ и любого $i$ верно, что $Z[i] \le i$.}
\pchoice{Для строки $S$ и любого $i$ верно, что $Z[i] \ge i$.}
\pchoice{Для строки $S$ и любого $i$ верно, что $\pi[i] \le i$.}
\pchoice{Для строки $S$ и любого $i$ верно, что $\pi[i] \ge i$.}
\pchoice{Для строки $S$ и любого $i$ верно, что $Z[i] \le Z[Z[i]]$.}
\pchoice{Для строки $S$ и любого $i$ верно, что $Z[i] \ge Z[Z[i]]$.}
\pchoice{Для строки $S$ и любого $i$ верно, что $Z[i] \le Z[\pi[i]]$.}
\pchoice{Для строки $S$ и любого $i$ верно, что $Z[i] \ge Z[\pi[i]]$.}
\pchoice{Для строки $S$ и любого $i$ верно, что $\pi[i] \le \pi[\pi[i]]$.}
\pchoice{Для строки $S$ и любого $i$ верно, что $\pi[i] \ge \pi[\pi[i]]$.}
\pchoice{$Z$-функция позволяет найти наибольшую общую подстроку двух строк $S$ и $T$ за $O(|S|+|T|)$.}
\pchoice{$Z$-функция позволяет найти все вхождения строки $S$ в текст $T$ за $O(|S|+|T|)$.}
\pchoice{$Z$-функцию для строки $S$ можно вычислить за $O(|S|)$.}
\end{tabular}



\item
Подстроку длины $m$ можно найти в строке длины $n$ за время:

\nopagebreak
\begin{tabular}{llllll}
\choice{$O(m^2n)$}\quad&
\choice{$O(mn^2)$}\quad&
\choice{$O(nn)$}\quad&
\choice{$O(m+n)$}\quad&
\choice{$O(n/m)$}
\end{tabular}

\item
Выберите верные утверждения:

\nopagebreak
\begin{tabular}{cp{12cm}}
\pchoice{Суффиксный массив (СМ) содержит номера суффиксов строки в порядке увеличения длины.}
\pchoice{СМ содержит номера суффиксов строки в лексикографическом порядке.}
\pchoice{СМ содержит номера префиксов строки в порядке увеличения длины.}
\pchoice{СМ содержит номера префиксов строки в лексикографическом порядке.}
\pchoice{СМ строки длины $n$ можно построить за время $O(n^2)$.}
\pchoice{СМ строки длины $n$ можно построить за время $O(n\log n)$.}
\pchoice{СМ строки длины $n$ можно построить за время $O(n/\log n)$.}
\pchoice{СМ строки длины $n$ позволяет искать в ней подстроку длины $m$ за $O(nm)$.}
\pchoice{СМ строки длины $n$ позволяет искать в ней подстроку длины $m$ за $O(n\log m)$.}
\pchoice{СМ строки длины $n$ позволяет искать в ней подстроку длины $m$ за $O(m\log n)$.}
\pchoice{СМ строки длины $n$ позволяет искать в ней подстроку длины $m$ за $O(m + \log n)$.}
\pchoice{СМ строки длины $n$ позволяет искать в ней подстроку длины $m$ за $O(n)$.}
\pchoice{СМ строки длины $n$ позволяет искать в ней подстроку длины $m$ за $O(m)$.}
\end{tabular}


\item
Укажите, для каких из приведенных задач вы знаете алгоритм решения за полиномиальное время.
Укажите его асимптотику.

\nopagebreak
\begin{tabular}{|p{8cm}|l|}
\hline
\bf Задача&\bf Время работы алгоритма\\
\hline
Поиск максимального паросочетания в двудольном графе&\\
\hline
Поиск паросочетания минимального веса в двудольном графе (задача о назначениях)&\\
\hline
Поиск минимального вершинного покрытия в двудольном графе&\\
\hline
\end{tabular}

\pagebreak
\item
Как будут выглядеть массивы $a$ порядка суффиксов длины $L$ и массив $c$ цветов при построении
суффиксного массива для циклической строки: <<ababaababba\$>> при $L = 4$?.
\vspace{7cm}

\item
Постройте суффиксный массив для строки: <<ababaababba>>  и посчитайте массив LCP.
\vspace{7cm}
\pagebreak

\item
Выберите истинные утверждения:

\nopagebreak
\begin{tabular}{cp{12cm}}
\pchoice{Можно проверить непустоту пересечения многоугольника с прямой за $O(\log n)$.}
\pchoice{Можно проверить непустоту пересечения выпуклого многоугольника с прямой за $O(\log n)$.}
\pchoice{Можно проверить непустоту пересечения многоугольника с прямой и, если оно непусто, 
            найти точку пересечения за $O(\log n)$.}
\pchoice{Можно проверить непустоту пересечения выпуклого многоугольника с прямой и, если 
            оно непусто, найти точку пересечения за $O(\log n)$.}
\pchoice{Можно проверить непустоту пересечения многоугольника с прямой за $O(n)$.}
\pchoice{Можно проверить непустоту пересечения выпуклого многоугольника с прямой за $O(n)$.}
\pchoice{Можно проверить непустоту пересечения многоугольника с прямой и, если оно непусто, 
            найти точку пересечения за $O(n)$.}
\pchoice{Можно проверить непустоту пересечения выпуклого многоугольника с прямой и, если 
            оно непусто, найти точку пересечения за $O(n)$.}
\end{tabular}

\item
    Постройте двудольный граф, в котором 8 вершин, 10 ребер и максимальное паросочетание имеет
    размер 2. Отметьте множества $L^+$, $L^-$, $R^+$ и $R^-$.
    \vspace{5cm}
\pagebreak
    
\item
Постройте автомат Ахо-Корасик для строк <<ab>>, <<b>>, <<abba>> и <<ba>>.
Приведите также суффиксные ссылки.
\vspace{8cm}    

\item
Посчитайте итоговую ассимптотику следующую функций
\begin{itemize}
\item
$T(n) = 2 T(\frac{n}{2}) + 1$
\vspace{1cm}
\item
$T(n) = 3 T(\frac{n}{2}) + n^2$
\vspace{1cm}
\item
$T(n) = 5 T(\frac{n}{2}) + n$
\vspace{1cm}
\item
$T(n) = 2 T(n - 1) + 1$
\vspace{1cm}
\end{itemize}
    
\end{enumerate}

\end{document}
