\documentclass[]{article}

\usepackage[utf8]{inputenc}
\usepackage[russian]{babel}
\usepackage{titlesec}
\usepackage[T2A]{fontenc}
\usepackage{amsmath, amssymb}
\usepackage[left=2cm,right=2cm,top=2cm,bottom=2cm,bindingoffset=0cm]{geometry}
\usepackage{setspace}
\usepackage{amsfonts}


\begin{document}

\fontsize{12}{15pt}\selectfont
\pagenumbering{gobble}
\pagenumbering{arabic}

\begin{center}
\fontsize{16}{16pt} \selectfont
\bf{
Домашнее задание

Шефер Эрика

M3139
}
\end{center}
\paragraph{1.} 

\begin{enumerate}

\item Существует всего $\geq 5!$ перестановок из пяти элементов. Тогда по теореме о нижней оценке сортировки, использующей сравнения, если всего существует $n$ перестановок исходного массива, то потребуется $\geq \log(n!)$ сравнений для его сортировки (высота двоичного дерева сравнений). 

Тогда, если $p$ - необходимое количество сравнений, то:

\begin{center}
$p \geq \log(5!)$, т.к. $p \in \mathbb{N} \Rightarrow 2^p \geq 5! \Rightarrow p \geq 7$.
\end{center}

{\bf Необходимо минимум 7 сравнений.}

\item Пусть есть массив из пяти элементов,  значение $i-$го элемента массива: $a_i$. Для того чтобы отсортировать  массив за 7 сравнений, нужно:
\begin{enumerate}
\item упорядочить $a_1, a_2, a_3, a_4$ так, что $a_1 \leq a_2$ и $a_3 \leq a_4$, {\bf+ 2 сравнения};
\item cравнить $a_1, a_3$, пусть получили, что $a_1 \leq a_3$, обратный случай рассматривается аналогично, {\bf+1 сравнение};
\item зная, что $a_1 \leq a_3 \leq a_4$, вставим в массив $a_5$ так, чтобы порядок возрастания сохранялся, используя бинпоиск, {\bf+2 сравнения}, т.к всего 4 возможных позиции для встаки.
\item зная, что $a_1 \leq a_2$ , вставим в массив $a_2$ так, чтобы порядок возрастания сохранялся, используя бинпоиск, {\bf+2 сравнения}, т.к. всего 4 возможных позиции для вставки.
\end{enumerate}
{\bf Всего: 7 сравнений}
\end{enumerate}

\newpage

\paragraph{2.}
\begin{enumerate}
\item Заметим, что сортировка выбором на каждой итерации делать от 0 до 1 обмена. Всего итераций $n - 1$, где $n$ - длина массива.

Тогда пусть существует перестановка $A$ c индексами от 1 до $n$. Тогда построим граф: проведем ребра из $i$ в $A_i$ (этот граф можно интерпретировать так: вершина из которой исходит ребро - это число, а вершина в которую это ребро входит - индекс позиции, на которой должно стоять это число в упорядоченной перестановке).

Посмотрим, что происходит, когда сортировка выбором совершает обмен: 


одно число $A_i$ встало на свое место, тогда эта вершина вышла из цикла, в котором была, образовалась отдельно стоящая вершина c петлей (ребро, ведущее из вершины в саму себя).
\item цикл лишился одной вершины, но остался замкнутым, так как при обмене, если были ребра $j \rightarrow A_j,  i \rightarrow A_i, k \rightarrow A_k$, при этом $A_j = i, A_i = k$ (т.е существует пусть $A_j \rightarrow A_i \rightarrow A_k$), тогда $A_i$ встало на свое место, 
\end{enumerate}
